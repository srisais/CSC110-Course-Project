\documentclass[fontsize=11pt]{article}
\usepackage{amsmath}
\usepackage[utf8]{inputenc}
\usepackage[margin=0.75in]{geometry}
\usepackage{indentfirst}

\title{CSC110 Project Proposal: Secondary School Course Enrolment during the COVID-19 Pandemic}
\author{Billy Guo and Sridhar Sairam}
\date{Tuesday, 14 December 2021}

\begin{document}
\maketitle

\section*{Problem Description and Research Question}  % Brief problem description and research question. (300–400 words)

Ontario secondary schools typically run on a semester system, where the first semester is from September to January and the second semester is from February to June. During each semester, students typically take 4 courses; students can drop (unenroll from) a course before the middle of the semester if they no longer wish to take it.

The COVID-19 pandemic started around March 2020, when students started learning from home for the remainder of the year. Since this was before the middle of the semester, students still had the opportunity to drop courses. However, due to the pandemic, the Ontario government said that final grades for semester 2 courses could not be lower than students’ grades before the March break (when the learning from home started). In addition, the number of hours per week of learning a teacher could assign was drastically reduced, causing some courses to be unable to achieve the same amount of learning it would otherwise have.

With experience for each course going to be very different from usual and considering that their mark would not decrease, would students still drop courses that they did not want to take? If many students dropped courses due to the pandemic, we would see a significant drop in overall course enrolments for that year, when compared to previous years. If course enrolments were similar when compared to other years, it would mean that the pandemic seemingly had not immediately impacted course enrollment.

Being students that experienced this situation, we wondered whether other students dropped courses. Given this, the following question arose:
\textbf{Did the COVID-19 pandemic have an immediate effect on the 2019 - 2020 school year’s course enrollment numbers in a significant manner?}

\section*{Dataset Description}  % A description of at least one relevant dataset you have found. (~150 words)

The datasets used are a collection of .txt and .xlsx files. Each school year has its own pair of .txt and .xlsx files. Since the data in both file formats are the same, just formatted differently, we opted with using the .txt files.

The datasets were found on the Canadian government’s Open Government section. The datasets themselves originate from the Ontario government’s Data Catalogue.

The .txt file is organized in a similar manner to a CSV file, but with a delimiter of a vertical bar, “$\mid$”. The courses that the dataset contains are courses defined by the Ontario Ministry of Education, for grades 9 - 12. Each row contains the following information about one specific course: course code, course description (course title), grade level, pathway or destination, and the number of students in that course. We use all of the columns of data in our project, however, most of the information regarding the course is mainly used just for labelling purposes.

\section*{Computational Overview}  % A computational overview for your project. (300–500 words)

The code was broken into 3 files: \texttt{functions\_and\_classes}, \texttt{command\_line\_interface}, and \texttt{main}.
Computationally, \texttt{main} doesn't really do that much. It just imports and runs the necessary functions
to run the program.

\texttt{functions\_and\_classes} is the basis of all of the computations and analysis. Within the module,
The classes and functions have been divided into parts: "Classes", "Importing .txt files", "Text File Data Extraction",
"Course Searching and Related", and "Statistical Analysis".
The "Classes" section of the module contained all of the classes that were used to represent the data
from the database files. For this, a \texttt{Course} class was created. This represented an Ontario course.
There is also a class attribute that keeps a collection of all \texttt{Course} objects; this ensures that
there are no duplicate \texttt{Course} objects (as in objects that are equivalent, value wise, but are stored in different memory locations).
Building on top of the \texttt{Course} object, we have the \texttt{CourseEnrollment} object.
This encapsulates a \texttt{Course} object, and its respective enrollment values, for a given school year.
The \texttt{SchoolYearAllCourseEnrollments} class encapsulates all \texttt{CourseEnrollment} for a given school year.
Then, the most important class, the \texttt{CourseEnrollmentHistory} class. This object contains a
collection of \texttt{CourseEnrollment} objects for a specific courses, where each \texttt{CourseEnrollment} object
represents a different school year. This object will allow us to see how a course's enrollment numbers
change throughout its history (throughout the school years).

The "Importing .txt files" and "Text File Data Extraction" help sanitize the input from the database files,
then group everything together. There are many helper functions that all end up in the grand result of\\
\texttt{get\_course\_enrollment\_history\_for\_all\_courses\_in\_folder}.
Using this function, we can input a directory path, and using the helper functions, it is able to
go through each database file, create a respective\\ \texttt{SchoolYearAllCourseEnrollments} for each file,
and then merge them all together to get a master list of\\ \texttt{CourseEnrollmentHistory} objects.
This master list is a huge list that contains the enrollment history for all of the courses in the database folder.

The "Course Searching and Related" functions intended to be used by the functions in the \texttt{command\_line\_interface}
module, but due to issues (that I will discuss in later section), these functions ended up not being used.

The "Statistical Analysis" functions are used to plot the graphs, displaying a course's enrollment history.
Here, two functions are used to plot 2 graphs. One function displays the actual course data with its
polynomial regression alongside it, and the second one displays the residual/difference graph for the
enrollment data and polynomial regression.

The \texttt{command\_line\_interface} consists of functions intended to communicate with the user,
so that they can choose what courses they'd like to run.
Originally, this was intended to be a section for a GUI, using \texttt{tkinter}, but for unforseen
reasons (that I'll get into in a later section), I had to create a CLI instead. Although I've broken
the module into 2 parts, the second part, "Main Loop", is just a function that uses all of the
helper functions. Speaking of, the "Command Line Interface Helper" breaks down each step of the
interactive process. When interacting with the user, we want to make sure that 1 misinput didn't
end the whole program, so adding some input sanitization, we grouped each step with functions.
The major computational procceses here are: validating input, raising and catching possible errors,
and repeatedly asking for a valid input until the user provides a valid input.

During this computational phase, we used \texttt{numpy} to create a polynomial regression for the enrollment data.
Using the regression, we used \texttt{matplotlib} to plot it and the enrollment data.
Additionally, we used the os library to find out what files there were in the data folder.
Without the os library, more manual work would have to have been done in pre-processing the data.
This helped automate lots of the manual work, making the file easier to run.

\section*{Instructions: Obtaining Data Sets and Running the Program}

These instructions are also included in the \texttt{main.py} file.

Instructions:
\begin{itemize}
    \item A folder named \texttt{database\_files} needs to be created in the same folder the 3 program files are in.
    \item All database (.txt) files that need to be downloaded must be placed in this newly made folder (\texttt{database\_files}).
    \item (including .txt files in the \texttt{database\_files} folder that are not a part of the database may cause problems)
    \item The specific database files to be downloaded can be found in the module description of \texttt{main.py}, or below, formatted as a Python list of URL strings.

\begin{verbatim}
["https://files.ontario.ca/opendata/mdc_enrol_1112_en_supp_2.txt",
    "https://files.ontario.ca/opendata/mdc_enrol_1213_en_supp_2.txt",
    "https://files.ontario.ca/opendata/mdc_enrol_1314_en_supp_2.txt",
    "https://files.ontario.ca/opendata/mdc_enrol_1415_en_supp_2.txt",
    "https://files.ontario.ca/opendata/mdc_enrol_1516_en_supp.txt",
    "https://files.ontario.ca/opendata/mdc_enrol_1617_en_supp.txt",
    "https://files.ontario.ca/opendata/mdc_enrol_1718_en_supp.txt",
    "https://data.ontario.ca/dataset/83b53104-19f9-4a77-8e38-a56c5575ef1c/resource/
    f4e0fef6-61f9-47cd-ae69-353b90fe0d7e/download/mdc_enrol_1819_en_supp.txt"]
\end{verbatim}

    \item After this is done, the \texttt{main.py} file can be run.
    \item Via interaction with the console, the user will be able to plot graphs (that will open in a new window).

\end{itemize}

What to Expect:
\begin{itemize}
    \item You should see an introductory mesasge, followed by more text that checks for the files.
    \item Given that you have placed the files in the correct place, you will be able to now
    search for courses. Here, you will give 2 numbers (indices), that represent a range of courses.
    \item Typing two numbers (separated by a space) will output the list of courses with numbers
    within that respective range!
    \item Once you are done searching through the courses, you can input an empty string
    to select a specific course to graph.
    \item Using the same course numbers from before, you will input a course number for
    the course you'd like to graph.
    \item After asking how many years to predict, and the degree of the polynomial regression,
    it will graph the course enrollment data, the polynomial regression, and the residuals (differences) graph
    (for the polynomial and the actual data)
    \item Then, the program will end.
\end{itemize}

\section*{Changes Made between the Project Proposal and Final Submission}

Originally, we intended to use matplotlib to graph plots and tkinter to create a GUI
to open these plots. Unfortunately, do to an unforseen problem, we weren't able to do that.
We ended up creating a CLI, where the user now interacts via the console, to search and select
a course to see its course enrollment history, and see how well the polynomial regression predicts future enrolments.
This was the only major change that occured.

The bug was one that was really odd and unexpected.
When we started to write the GUI using tkinter, I started to map out a general interface.
Using \texttt{grid}, I had an \texttt{Entry} section on the top, which had a \texttt{Button} beside it
that would take in a file path (the file path of the data set folder).
Using this, we would show a \texttt{Listbox}, and have a \texttt{Scrollbar}
so that one could scroll through all of the courses, and select the one they'd like to graph.
Then, they could use buttons on the bottom to show the graphs.

It seems like there wouldn't be a problem. It started off smooth sailing, but then, creating a \texttt{StringVar} instance
to get the text from the \texttt{Entry} is where the problem occured. 
I had to use functions from the \texttt{functions\_and\_classes} module, so I used the wildcard import.
This also imported \texttt{matplot.pyplot}.
When I tried to get the string data from \texttt{Entry} using \texttt{StringVar.get}, I was always getting an empty string!
However, when I removed the import statement, the \texttt{StringVar.get} method would return the correct string.
After lots of trial and error debugging, I found out that when I imported \texttt{matplotlib},
the \texttt{StringVar.get} method was not working.
After some digging, my final guess is due to the fact that the \texttt{matplotlib} library has
a top level function called \texttt{get}, which may be interfering the \texttt{StringVar.get}.
This being a guess, I'm not 100\% certain about this.

\section*{Discussion}

The main question was: "\textbf{Did the COVID-19 pandemic have an immediate effect on the 2019 - 2020
school year’s course enrollment numbers in a significant manner?}"

We don't have enough data at the moment to directly see that the COVID-19 pandemic caused course course enrollments
to change. We might end up concluding that correlation is due to the COVID-19 pandemic, which might not be the case.
However, we can take a few courses as an example, and see how well some polynomial regressions
do in predicting the course enrollments for the 2019 - 2020 school year, given the previous years' enrollment data.
We could predict that if COVID-19 was the cause, how much did some course's enrollments change.

For the following courses analyses, we will be predicting 1 year, with a polynomial degree of 5 or 6.

% 1054 - PPL40
% 1053 - PPL30
A course that one may suspect to have lower enrollments would be a physical education course,
such as PPL30 and PPL40 (grade 11 and grade 12 courses, respectively).
(When we ran the code, it had a course number of 1054).
When the polynomial degree is 5, it massively underpredicted the 2019 - 2020 school year course enrollment for PPL30.
The same thing occured when the degree was 6.
For PPL40, we see, for both polynomial degrees of 5 and 6, that the polynomial accurately predicts the drop in enrollments.
Seeing this difference in the same type of course, It's tough to tell whether or not this was due to COVID-19.
Interestingly, PPL30's enrollment increased, while PPL40's enrollment decreased, when compared to the 2018 - 2019 school year.


% 449 - ENG4U
ENG4U (Grade 12 University Preparation English) is a course that many highschool students in the 12th grade take.
This course is a prerequisite for (essentially) every university program.
There doesn't seem to be a particularily large change in enrollment for the course in the 2019 - 2020 school year.
Both polynomial regressions of degree 5 and degree 6 have $R^2$ very close to 1.
As expected, a course that must be taken by a large portion of students doesn't seem to have been
impacted by the COVID-19 pandemic.

% 1091 - SPH4U
% 1072 - SBI4U
% 1077 - SCH4U
SPH4U (Grade 12 University Preparation Physics) seems to have a large drop in enrollments in the 2019 - 2020 school year.
Both polynomials of degree 5 and 6 misestimate the amount by a very large margin.
Both of the polynomials have very low (essentially 0) coefficients of determination!
This means that there was a sudden drop in enrollments, than what would have otherwise been expected (by the polynomials).
A similar drop in enrolments can be seen in other grade 12 science courses, Biology (SBI4U) and Chemistry (SHC4U).
With this data in hand, it seems more likely that COVID-19 caused some course enrollments to drop,
while it caused others to say the same, or even increase.
(Course numbers during our run: 1091 - SPH4U, 1072 - SBI4U, and 1077 - SCH4U)

% 1152 - TDJ4M
% 1333 - TMR4M
Interestingly for technology courses TDJ4M and TMR4M, there is actually in increase in enrollments
from the 2018 - 2019 school year to the 2019 - 2020 school year.
Although the polynomial for TDJ4M has a very low coefficient of determination, due to the wild variation
in course enrollments from 2014 - 2017, the polynomial for TMR4M has a very high coefficient of determination (~0.98)!
This means that some courses' enrolments didn't vary very much in the 2019 - 2020 school year, based
on the polynomial prediction.
(Course numbers during our run: 1152 - TDJ4M and 1333 - TMR4M)

% 964 - MCV4U
% 970 - MHF4U
Finally, looking at the 5th degree polynomial predictions for MCV4U (Calculus and Vectors) and MHF4U (Advanced Functions),
we see 2 very different results!
We see that the polynomial for MCV4U overestimates, while the actual data shows a large drop.
Similarly, MHF4U also has a large drop in enrolments, however, the polynomial predicts this!
This may just be due to the fact that there was another large drop in enrollments in the 2014 - 2015 school year,
and the drop in the 2019 - 2020 school year just coincidentally accurately predicted it.
(Course numbers during our run: 964 - MCV4U, 970 - MHF4U)

Overall, if we assume that COVID-19 had some impact on the course enrollment of Ontario courses,
we could conclude that some courses had a decline in course enrollments, while other courses had an increase.
We might guess that "harder" courses such as math and science courses suffered a drop in enrollments, while
"easier" courses such as physical education and technological education courses saw an increase in enrollments.
However, this is just a guess, with no significant data to back up the claim.

For further analysis, we could use the 2020 - 2021 course enrollment data, when it is released,
and see how that data lines up with the previous years' data!

Additionally, depending on the degree of the polynomial, or the number of years we are predicting,
our results as to how acurately the polynomial predicts the course enrollment may vary.
By using a high enough degree, the end behaviour (based on whether or not it's an odd/even degree function)
could misguide predictions, as the rate of change would increase significantly.

\section*{References}
% Course enrolment in secondary schools - https://open.canada.ca/data/en/dataset/83b53104-19f9-4a77-8e38-a56c5575ef1c
% (don't really need to include this at the moment) Second language course enrolment - https://open.canada.ca/data/en/dataset/4546116e-e67f-4dcc-b2d7-9a701bd3be2f

\noindent Education, \& Government of Ontario. (2020). \emph{Course enrolment in secondary schools.} Retrieved from \linebreak \indent https://open.canada.ca/data/en/dataset/83b53104-19f9-4a77-8e38-a56c5575ef1c.

\noindent Education, \& Government of Ontario. (2021). \emph{Second language course enrolment.} Retrieved from \linebreak \indent https://open.canada.ca/data/en/dataset/4546116e-e67f-4dcc-b2d7-9a701bd3be2f. 

% NOTE: LaTeX does have a built-in way of generating references automatically,
% but it's a bit tricky to use so we STRONGLY recommend writing your references
% manually, using a standard academic format like APA or MLA.
% (E.g., https://owl.purdue.edu/owl/research_and_citation/apa_style/apa_formatting_and_style_guide/general_format.html)

\end{document}
